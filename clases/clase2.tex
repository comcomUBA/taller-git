% !TeX document-id = {2870843d-1baa-4f6a-bd0a-a5c796104a32}
% !BIB TS-program = biber
% !TeX encoding = UTF-8
% TU Delft beamer template

\documentclass[]{beamer}
\usepackage[english]{babel}
\usepackage{csquotes}
\usepackage{calc}
\usepackage[absolute,overlay]{textpos}
\usepackage{graphicx}
\usepackage{subfig}
\usepackage{mathtools}
\usepackage{amsfonts}
\usepackage{amsthm}
\usepackage{comment}
\usepackage{siunitx}
\usepackage{MnSymbol,wasysym}
\usepackage{array}
\usepackage{emoji}

\setemojifont{TwemojiMozilla}

\setbeamertemplate{navigation symbols}{} % remove navigation symbols
\mode<presentation>{\usetheme[verticalbar=false]{tud}}

% BIB SETTINGS
\usepackage[
    backend=biber,
    giveninits=true,
    maxnames=30,
    maxcitenames=20,
    uniquename=init,
    url=false,
    style=authoryear,
]{biblatex}
\addbibresource{bibfile.bib}
\setlength\bibitemsep{0.3cm} % space between entries in the reference list
\renewcommand{\bibfont}{\normalfont\scriptsize}
\setbeamerfont{footnote}{size=\tiny}
\renewcommand{\cite}[1]{\footnote<.->[frame]{\fullcite{#1}}}
\setlength{\TPHorizModule}{\paperwidth}
\setlength{\TPVertModule}{\paperheight}

\newcommand{\absimage}[4][0.5,0.5]{%
	\begin{textblock}{#3}%width
		[#1]% alignment anchor within image (centered by default)
		(#2)% position on the page (origin is top left)
		\includegraphics[width=#3\paperwidth]{#4}%
\end{textblock}}

\newcommand{\mininomen}[2][1]{{\let\thefootnote\relax%
	\footnotetext{\begin{tabular}{*{#1}{@{\!}>{\centering\arraybackslash}p{1em}@{\;}p{\textwidth/#1-2em}}}%
	#2\end{tabular}}}}


\title{Taller de Git}
\author{ComCom}
\institute{DC - FCEyN - UBA}

% http://latexcolor.com/
\definecolor{lightseagreen}{rgb}{0.13, 0.7, 0.67}
\definecolor{tangelo}{rgb}{0.98, 0.3, 0.0}
\definecolor{git}{rgb}{0.94, 0.309, 0.2}

\setbeamercolor{structure}{fg=lightseagreen}

\definecolor{x11gray}{rgb}{0.75, 0.75, 0.75}
\newcommand{\code}[1]{\Colorbox{x11gray}{\lstinline{#1}}}

\newenvironment{ejercicio}[1]{
% \setbeamercolor{block title}{bg=tangelo, fg=white}
\begin{exampleblock}{#1}
}{
\end{exampleblock}
}

\newenvironment{resumen}[1]{
\setbeamercolor{block title}{bg=git, fg=white}
\begin{block}{#1}
}{
\end{block}
}

\newenvironment{comando}{
\setbeamercolor{block body}{bg=git, fg=white}
\begin{block}{}
\begin{center}
\LARGE
\begin{texttt}
}{
\end{texttt}
\end{center}
\end{block}
}

\begin{document}
\begin{frame}
  \titlepage
  \begin{figure}[ht]
      \begin{center}
          \includegraphics[height=1in]{images/logo-taller.png}
      \end{center}
  \end{figure}
\end{frame}

\begin{frame}[t]{Recapitulando}
    \pause
    \begin{comando}
            \texttt{git status}
    \end{comando}

    \vspace{.4em}

    \begin{columns}[t]
        \begin{column}{0.05\textwidth}

        \end{column}
        \begin{column}{0.90\textwidth}
            \pause
            \hspace{-.25em}\texttt{Changes to be commited:}

            \pause
            \begin{block}{La clase anterior, aprendimos cómo:}
                \begin{itemize}
                    \pause
                    \item Obtener una copia local de un repositorio (\texttt{git\alt<-5>{...}{ clone}}).
                    \pause\pause
                    \item Iniciar un repositorio vacío (\texttt{git\alt<-7>{...}{ init}}).
                    \pause\pause
                    \item Marcar cambios como preparados o \textit{staged} (\texttt{git\alt<-9>{...}{ add}}),
                    \pause\pause y confirmar estos cambios (\texttt{git\alt<-11>{...}{ commit}}).
                    \pause\pause
                    \item Ver el estado actual de nuestros cambios (\texttt{git\alt<-13>{...}{ status}}).
                    \pause\pause
                    \item Enviar nuestros cambios a un repositorio remoto (\texttt{git\alt<-15>{...}{ push}}) \pause\pause y
                    bajarnos los cambios a nuestro repositorio local (\texttt{git\alt<-17>{...}{ pull}}).
                    \pause\pause
                    \item Resolver los conflictos que pueden presentarse al trabajar de forma colaborativa.
                \end{itemize}
            \end{block}

        \end{column}
        \begin{column}{0.05\textwidth}

        \end{column}
    \end{columns}

\end{frame}

\begin{frame}[t]{Recapitulando}
\begin{comando}
    \texttt{git status}
\end{comando}
    
    \vspace{.4em}

    \begin{columns}[t]
        \begin{column}{0.05\textwidth}

        \end{column}
        \begin{column}{0.90\textwidth}
            \hspace{-.25em}\texttt{Untracked files:}

            \pause
            \begin{block}{Hoy vamos a ver:}
                \begin{itemize}
                    \item Comandos para mover y eliminar archivos.
                    \item Comandos para inspeccionar cambios anteriores.
                    \item Ramificaciones (o \textit{branches}).
                    \item Algunos comandos un poco más avanzados.
                    \item \textit{Bonus track}.
                \end{itemize}
            \end{block}

        \end{column}
        \begin{column}{0.05\textwidth}

        \end{column}
    \end{columns}

\end{frame}

\begin{frame}[t]{Otros comandos útiles}
    \begin{comando}
        git rm
    \end{comando}

    \uncover<2->{
        \begin{block}{}
            Permite borrar un archivo y marcar este cambio como \textit{staged}. La sintaxis es \texttt{git rm [archivo]}.
        \end{block}
    }

    \vspace{2em}

    \begin{comando}
        git mv
    \end{comando}

    \uncover<3->{
        \begin{block}{}
            Permite mover/renombrar un archivo y marcar este cambio como \textit{staged}. La sintaxis es \texttt{git mv [archivo] [nuevo nombre/ubicación]}.
        \end{block}
    }
\end{frame}

\begin{frame}[t]{Inspeccionando los cambios}
    \begin{comando}
        git diff
    \end{comando}

    \pause
    \begin{block}{}

        Muestra las diferencias entre el estado actual de los archivos y la última vez que hicimos \texttt{git add} (cambios marcados como \textit{staged}).

        \vspace{.4em}

        Si, en cambio, queremos ver las diferencias entre los cambios marcados como \textit{staged} y los
        que confirmamos en el último \textit{commit}, podemos usar \texttt{git diff --staged}.
    \end{block}

    \pause
    \begin{ejercicio}{Ejercicio}
        Modificar un archivo de algún repo de la clase pasada y ejecutar \texttt{git diff}.
    \end{ejercicio}
\end{frame}

\begin{frame}[t]{Viendo la historia de los \textit{commits}}
    \begin{comando}
        git log
    \end{comando}

    \pause
    \begin{block}{}
        Muestra el historial de \textit{commits}. Además de quién hizo cada cosa y cuándo.
    \end{block}

    \pause
    \begin{ejercicio}{Ejercicio}
        Ejecutar \texttt{git log} en algún repo de la clase pasada.
    \end{ejercicio}
\end{frame}

\begin{frame}[t]{Ramificaciones en Git}

    % Git es distribuido

    Una \textbf{rama} (o \textit{branch}) en Git representa una línea independiente de desarrollo.
    Al crear nuevas ramas, podemos pensar que nuestro proyecto diverge en dos distintos:
    los cambios que hagamos en uno no impactan al otro.

    \pause
    \vspace{0.5em}
    Un ejemplo visual:

    \begin{figure}[ht]
        \begin{center}
            \includegraphics[height=1.5in]{images/branch.pdf}
        \end{center}
    \end{figure}

\end{frame}

\begin{frame}{Conceptos claves}
        \begin{itemize}
            \item {\textit{main:} rama predeterminada que se crea automáticamente cuando se crea un repositorio.}
            \pause
            \item {\textit{origin:} nombre predeterminado que recibe el repositorio remoto principal contra el que trabajamos.}
            \pause
            \item {\textit{HEAD:} commit en el que está tu repositorio posicionado en cada momento.}
        \end{itemize} 
\end{frame}

\begin{frame}[t]{Creando ramas}
    \begin{comando}
        git branch
    \end{comando}

    \pause
    \begin{block}{}
        Para crear una rama nueva, podemos ejecutar \texttt{git branch [nombre de la rama]}.

        \vspace{.4em}

        Nótese que este comando no nos mueve a la nueva rama, solo la crea.

        \vspace{.4em}

        Para ver las ramas de nuestro repositorio local, podemos ejecutar \texttt{git branch}.
    \end{block}

    \pause
    \begin{ejercicio}{Ejercicio}
        Crear una rama llamada ``prueba'' en algún repo de la clase pasada.
    \end{ejercicio}

\end{frame}

\begin{frame}[t]{Cambiando de rama}
    \begin{comando}
        git checkout
    \end{comando}

    \pause
    \begin{block}{}
        Para cambiar de rama, podemos ejecutar \texttt{git checkout [nombre de la rama]}.
    \end{block}

    \pause
    \vspace{0.5em}
    Un ejemplo visual:
    \vspace{-1.6em}

    \only<3-3>{
        \begin{figure}[ht]
            \begin{center}
                \includegraphics[height=1.6in]{images/checkout-branch-0.pdf}
            \end{center}
            \caption{}
        \end{figure}
    }

    \only<4-4>{
        \begin{figure}[ht]
            \begin{center}
                \includegraphics[height=1.6in]{images/checkout-branch-1.pdf}
            \end{center}
            \caption{Acá cambiamos a la rama ``testing'', ejecutando \texttt{git checkout testing}.}
        \end{figure}
    }

    \uncover<5-5>{
        \begin{figure}[ht]
            \begin{center}
                \includegraphics[height=1.6in]{images/checkout-branch-2.pdf}
            \end{center}
            \caption{Y los siguientes \textit{commits} serán agregados a la rama ``testing''.}
        \end{figure}
    }

\end{frame}

\begin{frame}[t]{Fusionando ramas}
    \begin{comando}
        git merge
    \end{comando}

    \pause
    \only<2-2>{
        \begin{block}{}
            Nos permite fusionar las historias de dos ramas distintas (podría haber conflictos).

            \vspace{.4em}

            La sintaxis es: \texttt{git merge [nombre de la rama a fusionar]}.

            \vspace{.4em}

            \textbf{Importante:} este comando fusiona la rama que le decimos
            \textbf{en la rama en la que estamos parados}.
        \end{block}
    }

    \pause
    Un ejemplo visual:
    \only<3-3> {
        \begin{figure}[ht]
            \begin{center}
                \includegraphics[height=1.5in]{images/merge-branch-0.pdf}
            \end{center}
            \caption{Antes del \textit{merge}.}
        \end{figure}
    }
    \only<4-4> {
        \begin{figure}[ht]
            \begin{center}
                \includegraphics[height=1.5in]{images/merge-branch-1.pdf}
            \end{center}
            \caption{Después de pararnos en la rama ``master'' (\texttt{git checkout master}) y haber fusionado la rama ``feature'' (\texttt{git merge feature}).}
        \end{figure}
    }
\end{frame}

\begin{frame}[t]{¡A practicar!}

    \pause
    \begin{block}{Pero antes de empezar...}
        Para hacer el siguiente ejercicio, vamos a trabajar con un \textbf{fork} de un repositorio.
        \pause\textit{¿Y eso?}

        \pause
        El concepto de \textit{fork} no es propio de Git. En el ámbito del desarrollo de software,
        el término se refiere a un proyecto que surge a partir de otro, y adopta un curso de
        desarrollo independiente.

        \pause
        Los servidores de Git, como GitLab, nos permiten hacer \textit{fork} de proyectos. Al hacer
        esto creamos una copia de un repositorio ajeno en nuestra propia cuenta, sobre la cual
        podemos trabajar libremente. Los cambios que hagamos en nuestra copia \textbf{no} impactarán
        en el repositorio original.
    \end{block}

    \pause
    \begin{center}
    \Large Ahora sí...
    \end{center}
\end{frame}

\begin{frame}[t]{¡A practicar!}
    \begin{ejercicio}{Ejercicio de a 2 máquinas (preferiblemente 2 personas): \emoji{alien} y \emoji{alien-monster}}
        \begin{enumerate}\begin{scriptsize}
            \pause
            \item \emoji{alien}: Hacer un \textit{fork} en su cuenta de \href{https://www.gitlab.com}{GitLab}
            de este repositorio: \url{https://gitlab.com/talleres-comcom/taller-git-ejercicio2}, y darle permiso a \emoji{alien-monster}
            para hacer \textit{push}.
            \pause
            \item \emoji{alien} y \emoji{alien-monster}: Obtener una copia local del repositorio de \emoji{alien}.
            \only<4>{\setcounter{enumi}{2}
            \item \emoji{alien} y \emoji{alien-monster}: El repo tiene una única rama, ``master'', donde van a encontrar un fragmento incompleto,
            de un cuento de Borges. Repartirse el trabajo: cada uno deberá completar un parrafo.

            \begin{figure}[ht]
                \begin{center}
                    \includegraphics[height=1.7in]{images/ejercicio-clase2-1.pdf}
                \end{center}
            \end{figure}
            }
            \only<5>{\setcounter{enumi}{3}\item \emoji{alien} y \emoji{alien-monster}: Crear, cada uno, una rama propia donde harán sus modificaciones. Posicionarse
            en la rama recién creada.

            \begin{figure}[ht]
                \begin{center}
                    \includegraphics[height=1.7in]{images/ejercicio-clase2-2.pdf}
                \end{center}
            \end{figure}
            }
            \only<6>{\setcounter{enumi}{4}\item \emoji{alien} y \emoji{alien-monster}: Completar la parte elegida del cuento, y hacer \textit{push} de estos
            cambios en el repositorio remoto.

            \begin{figure}[ht]
                \begin{center}
                    \includegraphics[height=1.7in]{images/ejercicio-clase2-3.pdf}
                \end{center}
            \end{figure}
            }
            \only<7>{\setcounter{enumi}{4}\item \emoji{alien} y \emoji{alien-monster}: Completar la parte elegida del cuento, y hacer \textit{push} de estos
            cambios en el repositorio remoto.

            \begin{figure}[ht]
                \begin{center}
                    \includegraphics[height=1.7in]{images/ejercicio-clase2-4.pdf}
                \end{center}
            \end{figure}
            }
            \only<8>{\setcounter{enumi}{5}\item \emoji{alien-monster}: Traer los cambios de la rama de \emoji{alien}.

            \begin{figure}[ht]
                \begin{center}
                    \includegraphics[height=1.7in]{images/ejercicio-clase2-7.pdf}
                \end{center}
            \end{figure}
            }
            \only<9>{\setcounter{enumi}{6}\item  \emoji{alien-monster}: Fusionar en la rama de \emoji{alien-monster} los cambios de \emoji{alien}. Enviar estos
            cambios al repositorio remoto.

            \begin{figure}[ht]
                \begin{center}
                    \includegraphics[height=1.7in]{images/ejercicio-clase2-8.pdf}
                \end{center}
            \end{figure}
            }
            \only<10>{\setcounter{enumi}{7}\item \emoji{alien}: Seguir los dos pasos anteriores, pero con los cambios de \emoji{alien-monster}.
            }
            \only<11>{
            \item \emoji{alien} y \emoji{alien-monster}: El repo tiene una única rama, ``master'', donde van a encontrar un fragmento incompleto,
            de un cuento de Borges. Repartirse el trabajo: cada uno deberá completar un parrafo.
            \item \emoji{alien} y \emoji{alien-monster}: Crear, cada uno, una rama propia donde harán sus modificaciones. Posicionarse
            en la rama recién creada.
            \item \emoji{alien} y \emoji{alien-monster}: Completar la parte elegida del cuento, y hacer \textit{push} de estos
            cambios en el repositorio remoto.
            \item \emoji{alien-monster}: Traer los cambios de la rama de \emoji{alien}.
            \item \emoji{alien-monster}: Fusionar en la rama de \emoji{alien-monster} los cambios de \emoji{alien}. Enviar estos
            cambios al repositorio remoto.
            \item \emoji{alien}: Seguir los dos pasos anteriores, pero con los cambios de \emoji{alien-monster}.
            }
        \end{scriptsize}\end{enumerate}
    \end{ejercicio}

\end{frame}

\begin{frame}[t]{Revirtiendo cambios}
    \begin{comando}
        git commit --amend
    \end{comando}

    \uncover<2->{
        \begin{block}{}
            Podemos usarlo para \textbf{arreglar}, por ejemplo, el mensaje del último \textit{commit} que hicimos: \texttt{git commit --amend -m [nuevo mensaje]}.
        \end{block}
    }

    \vspace{2em}

    \begin{comando}
        git revert
    \end{comando}

    \uncover<3->{
        \begin{block}{}
            Permite \textbf{revertir} exactamente los cambios introducidos por un \textit{commit}. Buscamos el \textit{hash} del \textit{commit} en cuestión
            usando \texttt{git log}, y luego ejecutamos \texttt{git revert [hash]}.
        \end{block}
    }
\end{frame}

\begin{frame}[t]{Otro comando útil}
    \begin{comando}
        git stash
    \end{comando}

    \pause
    \begin{block}{}
        Al ejecutar \texttt{git stash}, se guarda el estado actual de los archivos modificados y nos deja el directorio limpio.

        \vspace{0.5em}

        Para volver a mostrar los cambios guardados ejecutamos \texttt{git stash apply}.
    \end{block}
\end{frame}

\begin{frame}[t]{Extras: Ignorando archivos}

    \begin{block}{El archivo .gitignore}
      Es común tener archivos generados automáticamente que no queremos agregar al repositorio. Por ejemplo, archivos compilados: .pdf, .exe, .log, .pyc, etc. Sin embargo, es bastante molesto verlos todo el tiempo al ejecutar \texttt{git status}.

      Para arreglar esto podemos crear un archivo especial llamado \texttt{.gitignore}, que le indica a Git qué archivos \textbf{ignorar} por completo.
    \end{block}

    \pause
    \begin{resumen}{}
      Por ejemplo, para ignorar todos los archivos con extensión \textit{.pyc}:
      \begin{enumerate}
        \item Crear un archivo llamado \texttt{.gitignore} en el directorio principal del proyecto.
        \item Adentro escribir: \textit{*.pyc}
      \end{enumerate}
      Más ejemplos de \texttt{.gitignore}: \url{https://github.com/github/gitignore}.
    \end{resumen}


\end{frame}

\begin{frame}[t]{Extras: Más comandos}

    \begin{itemize}
        \item \texttt{git fetch [remote repository]}: Para traer todos los datos de un repositorio remoto.
        \item \texttt{git reset}: Permite deshacer cambios; sirve para revertir modificaciones en el área de trabajo, pero también puede eliminar por completo \textit{commits} anteriores. ¡Usar con mucho cuidado!
        \item \texttt{git rebase [branch]}: Aplica todos los \textit{commits} que difieren entre una rama y aquella en la que estamos parados. Así podemos incorporar cambios realizados en otras ramas manteniendo lineal el historial de la rama actual.
        \item \texttt{git blame [archivo]}: Para ver qué \textit{commit} modificó por última vez cada línea de un archivo, y quién fue su autor. ¡Así, podemos saber a quién \textit{culpar} cuando haya problemas!
        \item \texttt{git bisect}: Encuentra cuál fue el \textit{commit} que introdujo cierto error, haciendo búsqueda binaria en el historial de \textit{commits}.
    \end{itemize}

\end{frame}

\begin{frame}[t]{Extras: Servidores}

    \begin{figure}[ht]
        \begin{center}
            \includegraphics[height=0.7in]{images/github.png}
        \end{center}
    \end{figure}

    \begin{figure}[ht]
        \begin{center}
            \includegraphics[height=0.5in]{images/bitbucket.png}
        \end{center}
    \end{figure}

    \begin{figure}[ht]
        \begin{center}
            \includegraphics[height=0.7in]{images/gitlab.png}
        \end{center}
    \end{figure}

\end{frame}

\begin{frame}[t]{Bibliografía}

    \begin{itemize}
        \item Git Community book, disponible online y en español: \url{https://git-scm.com/book/es/v2}
        \item \texttt{git help [command]} para ver la documentación de cualquier comando de Git.
        \item A visual Git reference: \url{http://marklodato.github.io/visual-git-guide/index-es.html}
    \end{itemize}

\end{frame}

\begin{frame}{Feedback}

    \begin{center}
        \includegraphics[height=1.5in]{images/octocat-comcom.pdf}
    \end{center}

    \begin{block}{¡Nos interesa saber que les pareció el taller!}
        Pueden completar el formulario: \url{https://bit.ly/git-comcom-feedback-1c2024}

        \vspace{0.5em}

        También nos pueden tirar ideas para futuros talleres.
    \end{block}

\end{frame}

\end{document}