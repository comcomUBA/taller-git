\begin{frame}[fragile]{Configuraciones iniciales 1/3}

    \begin{block}{Tu identidad}
        Es importante establecer nuestro \textbf{nombre y email} en nuestro repositorio, ya que estos van a ir asociados con los cambios que hagamos:

        \vspace{0.5em}

        \texttt{git config --global user.name "Guybrush Threepwood"}

        \texttt{git config --global user.email guybrush@example.com}
    \end{block}

    \begin{block}{Modo de pull}
        Ejecuten esto (es una mauskiherramienta que nos ayudara mas tarde)
        \texttt{git config --global pull.rebase false}
    \end{block}

    \begin{block}{Config global vs local}
    También se puede omitir el flag \texttt{--global} para que las credenciales apliquen solo al repositorio en el que estamos trabajando.
    \end{block}

\end{frame}
\begin{frame}[fragile]{Configuraciones iniciales 2/3}

	\begin{block}{Clave SSH}
		Para poder trabajar cómodamente con repositorios Git que estén en Internet (GitHub, Bitbucket, GitLab, etc.), podemos configurar una \textbf{clave SSH} que nos identifique con el servidor que estemos usando. De esta manera, puede reconocer desde que usuario vienen los pedidos (para revisar permisos, etc...).

        %De esta manera, cuando tengamos un repositorio en Gitlab vamos a poder conectarlo con nuestro local.
	\end{block}
\end{frame}

\begin{frame}{Configuraciones iniciales 3/3}
  \begin{block}{Creando una clave nueva}
		Abrimos la terminal en el HOME y ejecutamos \textit{cd .ssh} y luego \textit{ssh-keygen -f id\_rsa}. Le damos \textit{Enter} a todo hasta que termine el proceso.
	\end{block}
  \pause
  \begin{block}{Subiendo la clave a GitLab}
		\begin{itemize}
		\item En la terminal, ejecutamos \textit{cat id\_rsa.pub} y copiamos todo lo que aparezca.
      \item Abrimos GitLab, vamos a Profile Settings, SSH Keys.
      \item Pegamos lo que tenemos copiado en el campo \textit{Key}.
      \item Ponemos para que expire mañana y le damos al boton \textit{Add key}.
		\end{itemize}
	\end{block}
    \pause
    \begin{block}{Funcionó?}
        Pueden probar que la clave quedó funcionando usando el siguiente comando: \textit{ssh git@gitlab.com}. Deberían ver su username de GitLab.
    \end{block}

\end{frame}
