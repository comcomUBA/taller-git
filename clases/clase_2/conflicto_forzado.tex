\begin{frame}{Git no decide nada}
    \begin{block}{DISCLAIMER - Mundos paralelos}
        Nos vamos a encontrar con varias situaciones donde git tiene que resolver que hacer frente a dos versiones "recientes" de un algo (archivo, repo, etc). Por ejemplo, repo local vs repo remoto, archivo commiteado vs archivo pusheado (por otra persona). Esto es complicado!
    \end{block}

    \pause

    En la clase anterior, mencionamos que git es muy sencillo y no toma decisiones por nosotros. Vamos a ver un ejemplo claro de esto.

\end{frame}

\begin{frame}[t]{Mandando fruta}

    \begin{ejercicio}{Ejercicio de a 2 personas en máquinas: \emoji{grapes} y \emoji{pear}}
        \begin{enumerate}\begin{small}
            \pause
            \item \emoji{grapes}: crear un repositorio nuevo en \href{https://www.gitlab.com}{GitLab},
            y darle permiso a \emoji{pear} para hacer \textit{push}.
            \pause
            \item \emoji{grapes} y \emoji{pear}: obtener una copia local del repositorio usando \textit{git clone}.

            \pause
            \item \emoji{grapes}:
            crear un archivo \textit{ensalada.txt} con una linea que diga "bowl". aca va a ir la ensalada. commitear y pushear.
            \item \emoji{pear}: clonar el archivo \textit{ensalada.txt}.
            \pause
            \item \emoji{grapes} y \emoji{pear}: hagan \textit{commits} (1 cada) agregando algunos nombres de frutas a su ensalada en la misma linea que "bowl".
            \pause
            \item \emoji{grapes}: hacer \textit{push} con sus cambios.
            \pause
            \item \emoji{pear}: intentar hacer \textit{push} de los cambios al repositorio remoto. ¿Qué pasó?
            \pause
            \item \emoji{pear}: bajarse los cambios del repositorio remoto con \textit{pull}. ¿Anduvo?
            \pause
            \item \emoji{pear}: resolver el conflicto (!) y pushear
            \pause
            \item \emoji{grapes}: bajarse los nuevos cambios.
            \pause
            \item \emoji{grapes} y \emoji{pear}: repetir estos pasos con los roles invertidos si aun no lo hicieron.
        \end{small}\end{enumerate}
    \end{ejercicio}

\end{frame}
