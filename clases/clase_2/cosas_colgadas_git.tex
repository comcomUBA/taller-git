\begin{frame}[t]{Recapitulando}
    \begin{block}{La clase anterior, aprendimos cómo:}
        \begin{itemize}
        \pause
            \item Iniciar un repositorio vacío (\texttt{git\alt<-2>{...}{ init}}).
            \pause\pause
            \item Marcar cambios como preparados o \textit{staged} (\texttt{git\alt<-4>{...}{ add}}), y confirmar estos cambios (\texttt{git\alt<-5>{...}{ commit}}).
            \pause\pause\pause
            \item Ver el estado actual de nuestros cambios (\texttt{git\alt<-7>{...}{ status}}).\pause
        \end{itemize}
    \end{block}
\end{frame}


\begin{frame}[t]{Otros comandos útiles}
    \begin{comando}
        git rm
    \end{comando}

    \uncover<2->{
        \begin{block}{}
            Permite borrar un archivo y marcar este cambio como \textit{staged}. La sintaxis es \texttt{git rm [archivo]}.
        \end{block}
    }

    \vspace{2em}

    \begin{comando}
        git mv
    \end{comando}

    \uncover<3->{
        \begin{block}{}
            Permite mover/renombrar un archivo y marcar este cambio como \textit{staged}. La sintaxis es \texttt{git mv [archivo] [nuevo nombre/ubicación]}.
        \end{block}
    }
\end{frame}


\begin{frame}[t]{Inspeccionando los cambios}
    \begin{comando}
        git show
    \end{comando}

    \uncover<2->{
        \begin{block}{}
            Muestra el contenido del último commit. Esto incluye tanto el contenido (adiciones, borrados, etc) como metadatos (fecha, autor).

            Podemos pedirle que nos muestre el contenido de algún contenido específico puntual.
        \end{block}
    }

    \begin{comando}
        git diff
    \end{comando}

    \uncover<3->{
        \begin{block}{}
    
        Muestra las diferencias entre el estado actual de los archivos y el último commit.

        \vspace{.4em}

        Si, en cambio, queremos ver las diferencias entre los cambios marcados como \textit{staged} y los
        que confirmamos en el último \textit{commit}, podemos usar \texttt{git diff --staged}.
        \end{block}
    }
\end{frame}

\begin{frame}[t]{Viendo la historia de los \textit{commits}}
    \begin{comando}
        git log
    \end{comando}

    \pause
    \begin{block}{}
        Muestra el historial de \textit{commits}. 
    \end{block}

    \pause
    \begin{ejercicio}{Ejercicio}
        Ejecutar \texttt{git log} en algún repo de la clase pasada.
    \end{ejercicio}

    % mostrar algunas flags para logear distintas cosas
    % git log -p
\end{frame}
