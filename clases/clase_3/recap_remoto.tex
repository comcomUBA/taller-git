\begin{frame}[t]{Repaso: obteniendo un repositorio Git de otro lado}
    \begin{comando}
        git clone
    \end{comando}

    \begin{block}{}
        Para obtener una \textbf{copia local} de un repositorio existente en algún servidor,
        utilizamos el comando \texttt{git clone [URL]}. (hagan esto en el HOME, no en .ssh!)

        En el caso de haber creado el repositorio con \textit{git init}, esto no es necesario. El comando \textit{clone} es como hacer \textit{init + remote add + pull}.
    \end{block}
    \begin{block}{}
        Importante: \textit{git clone [URL]} crea un nuevo directorio con el
        repositorio clonado. Debemos hacer \textit{cd <repo>} luego de clonarlo!
    \end{block}
\end{frame}

\begin{frame}[t]{Repaso: enviando cambios}
    \begin{comando}
        git push
    \end{comando}

    \pause
    \begin{block}{}
        Para enviar los cambios \textbf{desde nuestro repositorio local a algún
        repositorio remoto}, ejecutamos \texttt{git push}.
    \end{block}

    % \begin{ejercicio}{Ejercicio}
    %     Si tienen acceso al repo de la clase pasada, usen ese. Si no, pueden crear un nuevo usando los comandos que ya vimos. Asegúrense de hacer algún \textit{commit}. Luego hagan \textit{push}.
    % \end{ejercicio}
\end{frame}


\begin{frame}[t]{Repaso: recibiendo cambios}
    \begin{comando}
        git pull
    \end{comando}

    \begin{block}{}
        Para traer cambios \textbf{desde un repositorio remoto a nuestro repositorio local},
        ejecutamos \texttt{git pull}.
    \end{block}
\end{frame}

\begin{frame}[t]{Repaso: repositorios remotos}
    \begin{comando}
        git remote
    \end{comando}

    \vspace{0.5em}
    \pause
    \begin{block}{Ver los repositorios remotos asociados}
        Ejecutamos \texttt{git remote -v}. Si no hay ningun output, es que el repo no tiene ningún remote. Es decir, no apunta a ningún otro repo fuera del local.
    \end{block}

    \pause
    \begin{block}{Agregar un repositorio remoto}
        Ejecutamos \texttt{git remote add origin [URL]}. %chequear esto si funciona
    \end{block}

    \pause
    \begin{block}{git clone ya lo hizo}
        Si tenemos un repo que iniciamos con \texttt{git clone}, ya tenemos
        configurado el \texttt{remote} llamado \texttt{origin} automaticamente.
    \end{block}
\end{frame}

