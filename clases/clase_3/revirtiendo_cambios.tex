
\begin{frame}[t]{Revirtiendo cambios 1/2}
    \begin{comando}
        git commit --amend
    \end{comando}

        \begin{block}{}
            Podemos usarlo para \textbf{arreglar}, por ejemplo, el mensaje del último \textit{commit} que hicimos: \texttt{git commit --amend -m [nuevo mensaje]}.
            También sirve para añadir archivos al último commit que nos hayamos olvidado.
        \end{block}
        \pause
        \begin{ejercicio} {Ejemplo:}
        \texttt{git commit -m 'Commit inicial'}\\
        \texttt{git add archivo-olvidado}\\
        \texttt{git commit --amend}
        \end{ejercicio}
\end{frame}
\begin{frame}[t]{Revirtiendo cambios 2/2}
    \begin{comando}
        git revert
    \end{comando}

        \begin{block}{}
            Permite \textbf{revertir} exactamente los cambios introducidos por un \textit{commit}. Buscamos el \textit{hash} del \textit{commit} en cuestión
            usando \texttt{git log}, y luego ejecutamos \texttt{git revert [hash]}. También podemos deshacer uno o varios commits relativos al HEAD.
        \end{block}
    \begin{ejercicio}{Ejemplos:}
        \texttt{git revert HEAD\sim 3}\\
        Revierte los cambios del 4to commit anterior al HEAD. Crea un commit nuevo con los cambios.\\
        \pause
        \texttt{git revert -n master\sim 5..master\sim 2}\\
        Revierte los cambios del 5to al 3er commit en la rama master (incluído). La flag -n hace que no cree un commit nuevo con los cambios.
    \end{ejercicio}
    
\end{frame}

\begin{frame}[t]{Otro comando útil}
    \begin{comando}
        git stash
    \end{comando}

    \pause
    \begin{block}{}
        Al ejecutar \texttt{git stash}, se guarda el estado actual de los archivos modificados y nos deja el directorio limpio.

        \vspace{0.5em}

        Para volver a mostrar los cambios guardados ejecutamos \texttt{git stash apply}.
    \end{block}
\end{frame}

\begin{frame}[t]{Extras: Ignorando archivos}

    \begin{block}{El archivo .gitignore}
      Es común tener archivos generados automáticamente que no queremos agregar al repositorio. Por ejemplo, archivos compilados: .pdf, .exe, .log, .pyc, etc. Sin embargo, es bastante molesto verlos todo el tiempo al ejecutar \texttt{git status}.

      Para arreglar esto podemos crear un archivo especial llamado \texttt{.gitignore}, que le indica a Git qué archivos \textbf{ignorar} por completo.
    \end{block}

    \pause
    \begin{resumen}{}
      Por ejemplo, para ignorar todos los archivos con extensión \textit{.pyc}:
      \begin{enumerate}
        \item Crear un archivo llamado \texttt{.gitignore} en el directorio principal del proyecto.
        \item Adentro escribir: \textit{*.pyc}
      \end{enumerate}
      Más ejemplos de \texttt{.gitignore}: \url{https://github.com/github/gitignore}.
    \end{resumen}

\end{frame}

\begin{frame}[t]{Ejercicio}
    \begin{ejercicio}{}
\begin{enumerate}
    \item \emoji{dog}: Creá un archivo \textit{.md}, e incluílo en un commit nuevo, con un mensaje poco legible. Luego hagan push.
    \item \emoji{cat}: Creá un archivo \textit{.txt}, e incluílo en un commit nuevo, con un mensaje poco legible. Luego hagan push.
    \item \emoji{dog} y \emoji{cat}: pushear cambios y traerse los cambios de \emoji{cat} / \emoji{dog}.
    \item \emoji{dog} y \emoji{cat}: Cambienle el texto al commit de la otra persona con uno que se entienda más.
    \pause
\item \emoji{dog}: creá un archivo .gitignore, y agregá para que se ignoren todos los archivos \textit{.txt}. 
\item \emoji{cat}: creá un archivo .gitignore, y agregá para que se ignoren todos los archivos \textit{.md}. 
\item \emoji{dog} y \emoji{cat}: modifiquen los archivos de más arriba. Hagan sus commits, luego ambos pusheen y pullen los cambios del otro.
    ¿Qué paso?
\end{enumerate}
\end{ejercicio}
\end{frame}

\begin{frame}[t]{Extras: Más comandos}

    \begin{itemize}
        \item \texttt{git fetch [remote repository]}: Para traer todos los datos de un repositorio remoto.
        \item \texttt{git reset}: Permite deshacer cambios; sirve para revertir modificaciones en el área de trabajo, pero también puede eliminar por completo \textit{commits} anteriores. ¡Usar con mucho cuidado!
        \item \texttt{git rebase [branch]}: Aplica todos los \textit{commits} que difieren entre una rama y aquella en la que estamos parados. Así podemos incorporar cambios realizados en otras ramas manteniendo lineal el historial de la rama actual.
        \item \texttt{git blame [archivo]}: Para ver qué \textit{commit} modificó por última vez cada línea de un archivo, y quién fue su autor. ¡Así, podemos saber a quién \textit{culpar} cuando haya problemas!
        \item \texttt{git bisect}: Encuentra cuál fue el \textit{commit} que introdujo cierto error, haciendo búsqueda binaria en el historial de \textit{commits}.
    \end{itemize}

\end{frame}

\begin{frame}[t]{Extras: Servidores}

    \begin{figure}[ht]
        \begin{center}
            \includegraphics[height=0.7in]{images/github.png}
        \end{center}
    \end{figure}

    \begin{figure}[ht]
        \begin{center}
            \includegraphics[height=0.5in]{images/bitbucket.png}
        \end{center}
    \end{figure}

    \begin{figure}[ht]
        \begin{center}
            \includegraphics[height=0.7in]{images/gitlab.png}
        \end{center}
    \end{figure}

\end{frame}

\begin{frame}[t]{Bibliografía}

    \begin{itemize}
        \item Git Community book, disponible online y en español: \url{https://git-scm.com/book/es/v2}
        \item \texttt{git help [command]} para ver la documentación de cualquier comando de Git.
        \item A visual Git reference: \url{http://marklodato.github.io/visual-git-guide/index-es.html}
    \end{itemize}

\end{frame}

\begin{frame}{Feedback}

    \begin{center}
        \includegraphics[height=1.5in]{images/octocat-comcom.pdf}
        \includegraphics[height=1.5in]{images/feedback-24c2.png}
    \end{center}

    \begin{block}{¡Nos interesa saber que les pareció el taller!}
        Pueden completar el formulario: \url{https://bit.ly/git-comcom-feedback-2c2024}

        \vspace{0.5em}

        También nos pueden tirar ideas para futuros talleres.
    \end{block}

\end{frame}
