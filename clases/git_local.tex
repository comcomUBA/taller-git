
\begin{frame}{Motivación 1/2}

    \begin{figure}[ht]
        \begin{center}
            \includegraphics[height=1.5in]{images/caos.pdf}
        \end{center}
    \end{figure}

    \pause
    \begin{figure}[ht]
        \begin{center}
            \includegraphics[height=1.5in]{images/horror.png}
        \end{center}
    \end{figure}
\end{frame}
\begin{frame}{Motivación 2/2}

    \begin{block}{Trabajando en grupo}
        \begin{itemize}
            \item Enviar cambios por mail, o
            \pause
            \item Enviar archivos por Discord, o
            \pause
            \item Sincronizar cambios por Google Docs.
        \end{itemize}
    \end{block}

    \pause
    \begin{figure}[h]
        \begin{center}
            \includegraphics[height=1.5in]{images/horror.png}
        \end{center}
    \end{figure}

\end{frame}

\begin{frame}{¿Qué es un Sistema de Control de Versiones?}

	\begin{block}{}
 \begin{enumerate}
     \item Programas que permiten \textbf{manejar los cambios} en el código fuente de un proyecto a lo largo del tiempo.
     \item Llevan un \textbf{seguimiento} de las modificaciones que hacemos, y en caso de que nos equivoquemos, es posible volver atrás y comparar el código actual con versiones anteriores para ayudar a arreglar el error.
     \item Permiten que distintas personas modifiquen el código a la vez y \textbf{compartan los cambios}.
 \end{enumerate}

	\end{block}

 %   \pause
 %    \begin{resumen}{Es decir, permiten...}
 %        \begin{itemize}
 %            \item Arreglar \textit{accidentes} y volver a versiones anteriores del código.
 %            \item Compartir código con otras personas.
 %        \end{itemize}
	% \end{resumen}

\end{frame}

\begin{frame}{¿Qué es Git? 1/2}

	\begin{block}{}
 \begin{itemize}
     	\item Sistema de Control de Versiones \textbf{distribuido y de código abierto}.
  
        \item Con énfasis en la \textbf{performance} (para manejar proyectos muy grandes), \textbf{seguridad} y \textbf{flexibilidad}.
        
        \item Amplio conjunto de comandos que permiten realizar operaciones de alto y bajo nivel.

        \item Mantiene una copia local completa del proyecto.
 \end{itemize}

	\end{block}

    \begin{figure}[ht]
        \begin{center}
            \includegraphics[height=1.5in]{images/logo-git.pdf}
        \end{center}
    \end{figure}
\end{frame}

\begin{frame}{¿Qué es Git? 2/2}
    \begin{center}
        \begin{block}{GIT - the stupid content tracker}

            "git" can mean anything, depending on your mood.
            
             - \textbf{random three-letter combination that is pronounceable}, and not actually used by any common UNIX command.  The fact that it is a mispronunciation of "get" may or may not be relevant.\newline
             - \textbf{stupid}. contemptible and despicable. simple. Take your pick from the dictionary of slang.\newline
             - \textbf{"global information tracker"}: you're in a good mood, and it actually works for you. Angels sing, and a light suddenly fills the room. \newline
             - \textbf{"goddamn idiotic truckload of sh*t"}: when it breaks
        \end{block}
    \end{center}
    \pause
    \textit{ - Mensaje del commit de Linus Torvalds agregando el README al repositorio Git de Git (uf)}
    
\end{frame}

% revisar 
% se podría intercalar el status en todos los pasos
% \begin{frame}[t]{¿Cómo funciona Git?}
%     \begin{center}
%         \begin{block}{Sistema de \textit{Snapshots}}
%         \begin{enumerate}
%             \item Piensa a los datos como una serie de 'fotos' del sistema de archivos.
%             \item Con cada commit, o cada vez que guardas el estado del repo, saca una 'foto' del mismo y guarda una referencia a esa foto.
%             \item Si los archivos no cambian, no saca ninguna foto, solo mira la última.
            
%         \end{enumerate}

%         \end{block}

%         \includegraphics[width=4in]{images/snapshots.png}
%     \end{center}
    
%\end{frame}


% que es un repositorio?
% capaz piensan que un proyecto es un documento y no una carpeta

\begin{frame}{Repositorios}
    \begin{block}{¿Qué es un repositorio?}
    \begin{itemize}
        \pause
        \item La unidad básica donde guardaremos todos los elementos de nuestro proyecto.
        \pause
        \item Es un directorio o carpeta.
        \pause
        \item La idea es que nuestro proyecto, y todos los archivos que lo contengan, existan dentro de este repositorio.
        \pause
        \item Sabemos que una carpeta es un repositorio de git porque tiene dentro una subcarpeta \textit{.git}.
    \end{itemize}

    \end{block}
\end{frame}


\begin{frame}[t]{Creando un repositorio vacío}
    \begin{comando}
        git init
    \end{comando}

    \pause
    \begin{block}{}
        Crea un repositorio local vacío. Un lienzo en blanco, por así decirlo.
        \begin{enumerate}
            \item Nos paramos en el directorio que queremos convertir en un repositorio.
            \item Ejecutamos \texttt{git init}.
        \end{enumerate}
        Esto crea un subdirectorio \textit{.git} que tiene todos los archivos necesarios de Git.
    \end{block}
    \pause
    \begin{ejercicio}{Ejercicio}
    % decirles q los archivos y carpetas los hagan con bash
        Muévanse a la carpeta del ejercicio anterior. Dentro de ella, ejecuten \texttt{git init}. Recuerden los comandos de bash \textit{cd, ls}.
    \end{ejercicio}
\end{frame}
\begin{frame}[fragile, t]{¿Está preparado, confirmado o ninguna de las dos?}
    \begin{comando}
        git status
    \end{comando}
        \begin{block}{Output de ejemplo}
            \begin{center}
            \texttt{nothing to commit (create/copy files and use "git add" to track)}
            \end{center}
        \end{block}
    \pause
    \begin{ejercicio}{Ejercicio}
        Crear un nuevo archivo en nuestro repositorio y ejecutar \textit{git status} y vean como cambia el mensaje. Recuerden el comando \textit{touch}.
    \end{ejercicio}

\end{frame}

\begin{frame}[fragile, t]{¿Está preparado, confirmado o ninguna de las dos?}
    \begin{comando}
        git status
    \end{comando}
    \begin{block}{Output de ejemplo}
            \begin{center}
            \texttt{Untracked files:
  (use "git add <file>..." to include in what will be committed) \\archivo}
            \end{center}
        \end{block}
\begin{ejercicio}{Ejercicio}
    Ejecuten el comando que les sugiere \textit{git} para incluír el archivo que crearon recién. Hagan \textit{git status}.
\end{ejercicio}
\end{frame}

\begin{frame}[fragile, t]{¿Está preparado, confirmado o ninguna de las dos?}
    \begin{comando}
        git status
    \end{comando}
    \begin{block}{Output de ejemplo}
            \begin{center}
            \texttt{Changes to be committed:
  (use "git rm --cached <file>..." to unstage)
        \\new file:  archivo}
            \end{center}
        \end{block}
   \begin{block}{}
       Con esto vemos que \textit{git} ahora conoce al nuevo archivo, y lo va a tener en cuenta cuando hagamos un \textit{commit} proximamente.
   \end{block}     
\end{frame}

\begin{frame}{Estados de un archivo}

    \begin{block}{Los cuatro estados de los archivos}
            \begin{enumerate}
    \item Untracked = Sin seguimiento
    \item Unmodified = Sin modificar
    \item Modified = Modificado
    \item Staged, to be commited = Confirmado
    \end{enumerate}
    \end{block}
    \begin{center}
        \includegraphics[width=4in]{images/lifecycle.png}
    \end{center}    
\end{frame}


% es como un checkpoint
\begin{frame}[t]{Confirmando cambios}
    \begin{comando}
        git commit
    \end{comando}

    \pause
    \begin{block}{}
        Una vez que tenemos ciertos cambios marcados con \textit{add}, podemos confirmarlos
        ejecutando \texttt{git commit -m "mensaje"}.

        \vspace{0.5em}

        Donde \texttt{[mensaje]} es una breve descripción de los cambios que acabamos de confirmar.
        
        \pause
        Es como crear un checkpoint del estado del repositorio.
    \end{block}

    \pause
    \begin{ejercicio}{Ejercicio}
        Hagan el \textit{commit} que venían preparando y elijan un mensaje. Luego hagan \textit{git status}.
    \end{ejercicio}
\end{frame}



\begin{frame}[fragile]{Configuraciones iniciales}

    \begin{block}{Tu identidad}
        Es importante establecer nuestro \textbf{nombre y email} en nuestro repositorio, ya que estos van a ir asociados con los cambios que hagamos:

        \vspace{0.5em}

        \texttt{git config --global user.name "Guybrush Threepwood"}

        \texttt{git config --global user.email guybrush@example.com}


    También se puede omitir el flag \texttt{--global} para que las credenciales apliquen solo al repositorio en el que estamos trabajando.
        \end{block}
\end{frame}

\begin{frame}{Historial}
 \begin{comando}
     git log
 \end{comando}
 \pause
 \begin{block}{}
     Este comando nos sirve para ver el historial de cambios, o commits. Además nos dice quién hizo cada cosa, cuando, y su correo electrónico.
     
 \end{block}
    \begin{ejercicio}{Ejercicio}
        Fíjense quien es el autor de sus cambios anteriores... \emoji{face-with-tears-of-joy}
    \end{ejercicio}
\end{frame}

\begin{frame}[t]{¡No seas vago con los mensajes!}

    \begin{figure}[ht]
        \begin{center}
            \includegraphics[height=2in]{images/xkcd-git-commit.png}
        \end{center}
        \caption{Fuente: \url{https://xkcd.com/1296/}}
    \end{figure}

\end{frame}

\begin{frame}{Integrando los conceptos}
\begin{ejercicio} {Ejercicio integrador}
    \begin{enumerate}
        \item En el repositorio donde venían trabajando, hagan add y commit al archivo de Python que usaron para el ejercicio 1.
        \pause
        \item Usando nano, cambien de nuevo el contenido del archivo para que imprima el día de hoy.
        \pause
        \item Hagan de nuevo add y commit, con un mensaje que refleje su modificación.
        \pause
        \item Para ver el estado final de su repo con un gráfico, 
        
        hagan \textit{git log - -graph}
    \end{enumerate}
\end{ejercicio}
    
\end{frame}

