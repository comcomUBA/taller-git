\begin{frame}[t]{Recapitulando}
    \pause
    \texttt{git status}
    \vspace{.4em}

    \begin{columns}[t]
        \begin{column}{0.05\textwidth}

        \end{column}
        \begin{column}{0.90\textwidth}
            \pause
            \hspace{-.25em}\texttt{Changes to be commited:}

            \pause
            \begin{block}{La clase anterior, aprendimos cómo:}
                \begin{itemize}
                    \pause
                    \item Obtener una copia local de un repositorio (\texttt{git\alt<-5>{...}{ clone}}).
                    \pause\pause
                    \item Iniciar un repositorio vacío (\texttt{git\alt<-7>{...}{ init}}), \pause\pause
                    y vincularlo a un repositorio remoto (\texttt{git\alt<-9>{...}{ remote}}).
                    \pause\pause
                    \item Marcar cambios como preparados o \textit{staged} (\texttt{git\alt<-11>{...}{ add}}),
                    \pause\pause y confirmar estos cambios (\texttt{git\alt<-13>{...}{ commit}}).
                    \pause\pause
                    \item Ver el estado actual de nuestros cambios (\texttt{git\alt<-15>{...}{ status}}).
                    \pause\pause
                    \item Enviar nuestros cambios a un repositorio remoto (\texttt{git\alt<-17>{...}{ push}}) \pause\pause y
                    bajarnos los cambios a nuestro repositorio local (\texttt{git\alt<-19>{...}{ pull}}).
                    \pause\pause
                    \item Resolver los conflictos que pueden presentarse al trabajar de forma colaborativa.
                \end{itemize}
            \end{block}

        \end{column}
        \begin{column}{0.05\textwidth}

        \end{column}
    \end{columns}

\end{frame}

\begin{frame}[t]{Recapitulando}
    \texttt{git status}
    \vspace{.4em}

    \begin{columns}[t]
        \begin{column}{0.05\textwidth}

        \end{column}
        \begin{column}{0.90\textwidth}
            \hspace{-.25em}\texttt{Untracked files:}

            \pause
            \begin{block}{Hoy vamos a ver:}
                \begin{itemize}
                    \item Comandos para mover y eliminar archivos.
                    \item Comandos para inspeccionar cambios anteriores.
                    \item Ramificaciones (o \textit{branches}).
                    \item Algunos comandos un poco más avanzados.
                    \item \textit{Bonus track}.
                \end{itemize}
            \end{block}

        \end{column}
        \begin{column}{0.05\textwidth}

        \end{column}
    \end{columns}

\end{frame}
